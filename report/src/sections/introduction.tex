\section{Introduction}

In artificial intelligence, an agent is anything that can be viewed as perceiving its environment through sensors and acting upon that environment through actuators \parencite{russel2010}. However, such a broad definition is not particularly useful. Much more interesting is the idea of an agent (or system of agents) that operates with the objective to maximize its expected performance measure given its available knowledge and resources. Still, intelligent agents need not only to be rational, but also autonomous, that is, they should act independently by external control and learn to compensate for their partial or incorrect prior knowledge of the world.

Based on these ideas, the hereby presented project involved creating an autonomous and rational (multi-) agent system to play the game of Deliveroo. The task is inspired by the real-world problem of courier services, where parcels have to be delivered to different locations in a timely and efficient manner. In more details, the game is played on a two-dimensional map, where parcels are randomly generated and spawns with a random reward value that may decay over time. The objective of the agents is to collect and deliver parcels to the designated delivery locations in order to maximize their total reward.

To make the problem more challenging, the agents have to deal with a number of constraints and limitations. First, while the underlying mechanics of the game are known, the environment is only partially observable to the agents which can only perceive their surroundings within a certain radius. Second, the agents have to deal with the stochastic and dynamic nature of the game, as the parameters of the game (e.g. the number of parcels, their reward distribution, the movement speed of the agents) can change from one game to another. As an additional challenge, other agents can be present in the environment and compete for the same resources. Such agents can be either adversarial or cooperative, thus requiring the agents to adapt their strategies accordingly.

In the following sections, we describe the design and implementation of the implemented agent system. While the original project required the implementation of both a single-agent and a multi-agent system, here we focus on the latter as the single-agent system is a special case of the multi-agent system (no communication and no Hungarian matching). We will also present the results of the experiments conducted to evaluate the performance of our system. Finally, we will discuss the challenges and limitations of our approach, and suggest possible improvements for future work.